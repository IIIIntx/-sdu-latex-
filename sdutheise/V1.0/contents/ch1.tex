% !Mode:: "TeX:UTF-8"

\chapter{使用简介}
\echapter{Introduction}

\section{章节的修改和添加}
你可通过"chapter{}"和"\textbackslash echapter{}"来添加章节题目,并在SDUthesistemplate.tex文件的正文部分中启用。
通过这种方式,你可以灵活的修改章节,但请记住,务必去掉tex文件中注释的章节,否则会影响生成。

\section{摘要、致谢、附录和引用}
你可以在\textbackslash contents\textbackslash abstract.tex中编辑中英文摘要的相关内容。

你可以在\textbackslash contents\textbackslash acknowledgment.tex中编辑致谢内容。

你可以在\textbackslash contents\textbackslash appendix中编辑附录内容,如果你需要多个附录,可以在SDUthesistemplate.tex文件的附录部分
\textbackslash appendix进行\textbackslash input其他内容。
如果你需要在附录中附注核心代码内容,你可以直接查看当前appendix.tex中的内容并进行修改。

你可以在\textbackslash contents\textbackslash reference.bib中添加引用文献,请注意,在bib文件中的引用格式应使用bibtex格式,你可以在google scholar或
期刊文献出版社页面找到。
如果需要引用,请参考以下代码\cite{phattharasupakun2020impact}

\section{表格和图片的插入}
如果你需要插入表格,你可以使用https://www.tablesgenerator.com/该网站进行表格的预绘制,随后直接粘贴进入tex文件。

以下或许是一个可供你参考的样例
\begin{table}[htbp]
    \centering
\begin{tabular}{|p{3em} p{18em} p{3em} p{8em}|}
    \hline
    你 & 好 & 欢 & 迎\\ \hline
    你 & 使 & 用 & \LaTeX \\
    \hline
\end{tabular}
    \caption{注释表}
\end{table}

如果你要插入图片,单张图片插入你可以使用
\begin{figure}[htbp]
    \centering
    \includegraphics[width=0.8\textwidth]{figures/DECAF1.jpg}
    \caption{样例1}
\end{figure}
如果你要并列插入两张图片,你可以使用
\begin{figure}[htbp]
    \centering
    \subfloat[样例2.1]{\includegraphics[width=0.4\textwidth]{figures/fig11.png}}
    ~~
    \subfloat[样例2.2]{\includegraphics[width=0.4\textwidth]{figures/fig12.png}}
    \caption{样例2}
    \label{thisiisalabel}
\end{figure}

如果你需要一个公式,你可以使用
\begin{equation}
    SOH=\frac{C_{CUR}}{C_{RATED}}\time 100\%
\end{equation}
或是$\pi$ 或是$$SOH=\frac{C_{CUR}}{C_{RATED}}\time 100\%$$
对于双\$的公式,他不会产生标号。

\begin{equation}
    K=
    \left[
    \begin{array}{c c c}
        K_{s,s} & K_{s,t} \\
        K_{t,s} & K_{t,t} \\
    \end{array}
    \right]
\end{equation}

\begin{equation}
    \frac{\partial z^{(k)}}{\partial W^{(k)}}=
    \left[
        \begin{array}{c}
            \frac{\partial(\mathcal{W}_1(k)*n(k-1)+b(k))}{\partial W(k)} \\
            \dots \\
            \dots \\
            \dots \\
            \frac{\partial(\mathcal{W}_m(k)*n(k-1)+b(k))}{\partial W(k)} \\
        \end{array}
    \right]
\end{equation}

如果你需要进行举例,你可以使用:
\begin{enumerate}
    \item \textbf{我是1}
    
    你好1
    \item \textbf{我是2}
    
    你好2
    \item \textbf{我是3}
    
    你好3
\end{enumerate}

如果你在生成pdf的时候遇到了出现空白页的情况,请不要担心,这是由于默认双页的原因,你可以插入下列代码
\let\cleardoublepage\clearpage

对了,当你需要编译时,如果希望快速可以使用xelatex进行编译,因此我也建议你如果在VScode环境下使用,可以设置xelatex为首选项。
但此时编译的结果没有引文标号,如果你需要完整编译,呈现出引文标号时,你可以使用pdf->bib-pdf->pdf的方法编译。